\section*{Abstract}
We express thermodynamics as the rate language of a discrete tick medium,
where temperature is energy per entropy-tick and balance laws record local
updates. Standard relations are re-cast in this framework:
\(T=\dot Q/\dot S_{\mathrm{phys}}\) defines temperature near equilibrium, and
the entropy-balance law \(\partial_t s+\nabla\!\cdot\!\mathbf j_S=\sigma\ge0\)
governs transport and production.

Entropy flows through three measurable channels—conduction (diffusive),
convection (advective), and radiation (ballistic). A near-blackbody rule,
\(\dot S_{\mathrm{rad}}=\tfrac{4}{3}P/T\), serves as an entropy-throughput
meter in vacuum. Worked examples convert power to entropy and information
rates via \(k_B\ln2\).

Planck-unit checkpoints clarify the roles of \(c,\hbar,G,k_B\) inherited from
earlier papers and anchor \(\dot S_{\mathrm{phys}}=\dot Q/T\) as the operational
link between microscopic ticks and macroscopic thermodynamics. The tick
sequence sets time orientation, while the thermodynamic arrow arises from
\(\sigma\ge0\) and low-entropy initial conditions.
