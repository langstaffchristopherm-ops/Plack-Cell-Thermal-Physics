 \section*{Abstract}
We express thermodynamics as the rate language of the tick medium: temperature is energy per entropy-tick, and local balance laws are just update bookkeeping.
We recast thermodynamics as the natural “rate language” of a Planck-Cell medium in which physical change is organized by discrete ticks. Rather than proposing new laws, we make standard results operational in this medium: temperature becomes the update-rate ratio \(T=\dot Q/\dot S_{\mathrm{phys}}\) near equilibrium, and local entropy balance \(\partial_t s+\nabla\!\cdot\!\mathbf j_S=\sigma\) with \(\sigma\ge0\) governs production and transport.

We emphasize measurable channels of entropy flow. Conduction gives nonnegative production via Fourier’s law; convection carries entropy with bulk motion; and radiation behaves as a ballistic channel. A useful near-blackbody rule, \(\dot S_{\mathrm{rad}}=\tfrac{4}{3}P/T\), provides a simple entropy-throughput meter in vacuum.

Worked examples—heating water and the Sun’s luminosity—show how to convert power into entropy rates and, via \(k_B\ln 2\), into information rates (bits/s). A Planck-unit checkpoint clarifies roles of \(c,\hbar,G\) inherited from prior chapters and anchors \(k_B\) operationally through \(\dot S_{\mathrm{phys}}=\dot Q/T\), linking microscopic ticks to macroscopic thermodynamics.

Conceptually, ticks provide time orientation (before/after), while the thermodynamic arrow arises from \(\sigma\ge0\) for closed systems together with assumed low-entropy initial conditions (not derived here). The chapter fixes notation and units, stays within established thermodynamics, and sets up subsequent dynamics (e.g., electrodynamics in the Planck-Cell medium) that aim to close the remaining constants within the same rate-based framing.