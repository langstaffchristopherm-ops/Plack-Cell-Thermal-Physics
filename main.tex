\documentclass[11pt,oneside]{article}

% Centralized packages/macros
% ==============================
% format_macros.tex — PDF figs; single DOI macro; link colors; glossaries (no biblatex)
% ==============================

% ---- Release metadata (single source of truth) ----
\IfFileExists{release.tex}{\input{release.tex}}{}

% Fallbacks if release.tex is missing
\providecommand{\RepoDOI}{10.5281/zenodo.XXXXXXX}
\providecommand{\RepoCommit}{0000000}
\providecommand{\PaperVersion}{v0.0.0}
\providecommand{\PaperLicense}{(license not set)}

% Canonical DOI link helper
\newcommand{\RepoDOIlink}{\href{https://doi.org/\RepoDOI}{\nolinkurl{https://doi.org/\RepoDOI}}}

% --- Core packages (pdfLaTeX-safe) ---
\usepackage[utf8]{inputenc}
\usepackage[T1]{fontenc}
\usepackage{lmodern}
\usepackage{microtype}
\usepackage{amsmath,amssymb}
\usepackage{graphicx}
\usepackage{xcolor}
\usepackage{booktabs}
\usepackage{array}
\usepackage{threeparttablex}
\usepackage{siunitx}
\usepackage{xurl}
\usepackage{ragged2e}
\usepackage[english]{babel}
\usepackage[unicode]{hyperref}
\usepackage{cleveref}
\usepackage[most]{tcolorbox}
\usepackage[acronym]{glossaries} % optional
\usepackage{fancyhdr}
\usepackage{tabularx}
\usepackage[autostyle=true]{csquotes}
\usepackage[font=small,labelfont=bf]{caption}
\captionsetup{justification=raggedright,singlelinecheck=false}
\usepackage{placeins}
\usepackage{float}
% \usepackage{epigraph} % <-- comment out if not using epigraphs

% --- Column helpers ---
\newcolumntype{L}[1]{>{\RaggedRight\arraybackslash}p{#1}}
\newcolumntype{Y}{>{\RaggedRight\arraybackslash}X}

% --- Identity ---
\providecommand{\PaperTitleMain}{Planck-Cell --- Thermal Physics}
\providecommand{\PaperSubtitleMath}{XXXXXXXXXX}
\providecommand{\AuthorName}{Christopher Michael Langstaff}
\providecommand{\AuthorNameShort}{C.~Langstaff}
\providecommand{\AuthorORCID}{0009-0000-3960-773X}
\providecommand{\PaperLicense}{CC BY 4.0}
\providecommand{\RepoURL}{https://github.com/langstaffchristopherm-ops/Temporal_Relativity}

% --- Page style + spacing ---
\pagestyle{fancy}
\fancyhf{}
\lhead{\PaperTitleMain}
\rhead{\AuthorNameShort}
\cfoot{\thepage}
\setlength{\headheight}{14pt}
\renewcommand{\headrulewidth}{0.3pt}
\renewcommand{\footrulewidth}{0pt}
\setlength{\textfloatsep}{12pt plus 2pt minus 2pt}
\setlength{\intextsep}{10pt plus 2pt minus 2pt}
\emergencystretch=2em
\Urlmuskip=0mu plus 1mu

% --- PDF metadata ---
\hypersetup{
  pdftitle   = {XXXXXXXXXX --- XXXXXXXXXX},
  pdfauthor  = {\AuthorName},
  pdfsubject = {XXXXXXXXXX},
  pdfkeywords= {X,X,X,...},
  colorlinks=true, linkcolor=blue, citecolor=blue, urlcolor=blue
}

% --- Glossaries (optional) ---
\newif\ifwithglossary
\withglossarytrue
\ifwithglossary
  \makeglossaries
\fi

% --- Breakable code/paths ---
\urlstyle{tt}
\DeclareRobustCommand{\code}[1]{\nolinkurl{#1}}

% --- Safe figure include (placeholder if missing) ---
\newcommand{\safeincludegraphics}[2][]{%
  \IfFileExists{#2}{\includegraphics[#1]{#2}}{%
    {\setlength{\fboxsep}{0pt}\fbox{\rule{0pt}{0.6in}\rule{0.9\linewidth}{0pt}}}%
    \typeout{[safeincludegraphics] Missing figure: #2}}}

% --- Provenance line ---
\newcommand{\figprov}[4]{%
  \par\smallskip\noindent
  \begingroup\footnotesize\RaggedRight\sloppy\setlength{\emergencystretch}{12em}%
  \textit{Provenance:} src=\code{#1}; data=\code{#2}; commit=\code{#3}.%
  \par\endgroup
}

% --- Flat layout helpers ---
\providecommand{\src}[1]{src/#1}
\providecommand{\data}[1]{data/#1}
\providecommand{\fig}[1]{figs/#1}

% --- siunitx defaults + small table helper ---
\sisetup{
  reset-text-series = false, text-series-to-math = true,
  reset-text-family = false, text-family-to-math = true,
  group-minimum-digits = 4,
  round-mode = places, round-precision = 3
}
\newenvironment{smalltable}[1][]{\begin{threeparttable}[#1]\small}{\end{threeparttable}}

% --- Claims & Tests table (compact) ---
\newenvironment{claimstable}{%
  \par\smallskip\noindent
  \begingroup\footnotesize\sloppy\setlength{\emergencystretch}{2em}%
  \setlength{\tabcolsep}{2pt}%
  \renewcommand{\arraystretch}{1.07}%
  \begin{tabular}{@{}L{0.18\linewidth} L{0.34\linewidth} L{0.34\linewidth} L{0.08\linewidth}@{}}
  \toprule
  \textbf{Claim (Level)} & \textbf{Equation / Algorithm} & \textbf{Data \& Script} & \textbf{Pass} \\
  \midrule
}{%
  \\ \bottomrule
  \end{tabular}%
  \par\endgroup\smallskip
}

% --- Level tags ---
\newcommand{\Lzero}{\textsf{L0}}
\newcommand{\Lone}{\textsf{L1}}
\newcommand{\Ltwo}{\textsf{L2}}
\newcommand{\Lthree}{\textsf{L3}}
\newcommand{\Lfour}{\textsf{L4}}

% --- Non-Goals box ---
\newenvironment{nongoalsbox}{%
  \begin{tcolorbox}[colback=white,colframe=black!12,title=Non-Goals]
}{\end{tcolorbox}}

% ------------------------------
% Lists toggle (LoF/LoT)
% ------------------------------
\newif\ifShowLists
\ShowListstrue % set \ShowListsfalse to hide LoF/LoT

% ------------------------------
% Safe glossary gate — define ONCE
% ------------------------------
\makeatletter
\providecommand{\IfGlossariesEnabled}[1]{\@ifpackageloaded{glossaries}{#1}{}}
\makeatother


% Optional math + theorem layers (safe if absent/empty)
\IfFileExists{other_tex/math.tex}{\input{other_tex/math.tex}}{}
\IfFileExists{other_tex/theorems.tex}{\usepackage{epigraph}
% ==============================
% theorems.tex — theorem-like environments
% ==============================
\usepackage{amsthm}

\theoremstyle{plain}
\newtheorem{theorem}{Theorem}
\newtheorem{lemma}[theorem]{Lemma}
\newtheorem{proposition}[theorem]{Proposition}

\theoremstyle{remark}
\newtheorem*{remark}{Remark}


}{}

% -------------
% Metadata
% -------------
\title{\PaperTitleMain}
\ifdefined\Authors
  \author{\Authors}
\else\ifdefined\AuthorName
  \author{\AuthorName}
\else
  \author{Author Name}
\fi\fi
\date{\today}

\begin{document}
\maketitle

% Dedication inline under the title (no page break)
\IfFileExists{other_tex/dedication.tex}{%
  \vspace{0.75\baselineskip}\cleardoublepage
\begingroup
  \hypersetup{pageanchor=false}% prevent duplicate 'page.i' anchor
  \thispagestyle{empty}
  \vspace*{\fill}
  \begin{center}\itshape
  To the wanderers who learn, step by step, that the gifts they seek—courage, heart, and wisdom—were with them all along.
  \end{center}
  \vspace*{\fill}
  \clearpage
\endgroup%
}{}

% ----------------
% Abstract (single source of truth; add to ToC once)
% ----------------
\section*{Abstract}
\addcontentsline{toc}{section}{Abstract}
\IfFileExists{other_tex/abstract.tex}{We recast thermodynamics as the natural “rate language” of a Planck-Cell medium in which physical change is organized by discrete ticks. Rather than proposing new laws, we make standard results operational in this medium: temperature becomes the update-rate ratio \(T=\dot Q/\dot S_{\mathrm{phys}}\) near equilibrium, and local entropy balance \(\partial_t s+\nabla\!\cdot\!\mathbf j_S=\sigma\) with \(\sigma\ge0\) governs production and transport.

We emphasize measurable channels of entropy flow. Conduction gives nonnegative production via Fourier’s law; convection carries entropy with bulk motion; and radiation behaves as a ballistic channel. A useful near-blackbody rule, \(\dot S_{\mathrm{rad}}=\tfrac{4}{3}P/T\), provides a simple entropy-throughput meter in vacuum.

Worked examples—heating water and the Sun’s luminosity—show how to convert power into entropy rates and, via \(k_B\ln 2\), into information rates (bits/s). A Planck-unit checkpoint clarifies roles of \(c,\hbar,G\) inherited from prior chapters and anchors \(k_B\) operationally through \(\dot S_{\mathrm{phys}}=\dot Q/T\), linking microscopic ticks to macroscopic thermodynamics.

Conceptually, ticks provide time orientation (before/after), while the thermodynamic arrow arises from \(\sigma\ge0\) for closed systems together with assumed low-entropy initial conditions (not derived here). The chapter fixes notation and units, stays within established thermodynamics, and sets up subsequent dynamics (e.g., electrodynamics in the Planck-Cell medium) that aim to close the remaining constants within the same rate-based framing.
}{}

% =========================================================
% =====================  MAIN TEXT  =======================
% =========================================================

\section{Thermal Physics in the Planck-Cell Medium}
\subsection*{\textit{Why entropy per tick is the default language of change}}

% --- Hub/standalone-safe macro & notation guards ---
\providecommand{\Sent}{S_{\mathrm{ent}}}             % dimensionless entropy (nats)
% \providecommand{\chiRate}{\chi}                    % (removed to avoid symbol collision with \chi=S/t)
\newcommand{\Sphys}{S_{\mathrm{phys}}}               % entropy (J/K)
\newcommand{\dotSphys}{\dot{S}_{\mathrm{phys}}}      % entropy rate (J/K/s)

\section*{Scope}
Thermodynamics is often taught as the ``boiler room'' of physics—engines, refrigerators, heat transfer—while mechanics and fields get to be ``fundamental.'' In the Planck-Cell (S/t) framework, each tick \emph{permits} entropy exchange/production. Thermodynamics is therefore the \emph{default language of change}; mechanics and fields are corollaries of how entropy accumulates and flows \cite{callen1985thermodynamics}. Foundations of the tick-based framing appear in earlier parts of the program \cite{langstaff_zenodo_16908311}.

\emph{Scope guard.} Ticks supply \emph{time orientation} (before/after). Actual entropy growth follows from the local balance with nonnegative production in closed systems; cosmological low-entropy initial conditions are assumed, not derived here \cite{zeh2007arrow}.

% (Optional, put in your macros file to have a symbol for bits-entropy)
\newcommand{\Sbits}{S_{\mathrm{bits}}}

\subsection*{Notation \& units}
\begin{itemize}
  \item \textbf{Dimensioned entropy variables.}
  We use \(\Sphys\) (J/K) and \(\dotSphys\) (J/K/s) for all heat/temperature relations:
  \[
    \delta Q = T\,d\Sphys, \qquad \dotSphys = \frac{\dot Q}{T}.
  \]

  \item \textbf{Dimensionless counting.}
  We use \(\Sent\) (nats), with the rate identity
  \[
    \dot{\Sent} = \frac{\dotSphys}{k_B}.
  \]
  Conversion to bits uses
  \[
    S_{\text{bits}} = \frac{\Sent}{\ln 2}
    \quad\text{(or }\Sbits = \Sent/\ln 2\text{ if you enabled the macro).}
  \]

  \item \textbf{Not used here.}
  We \emph{do not} use the mechanics/gravitation potential-like field \(\chi = S/t\) from earlier chapters.
\end{itemize}


\paragraph{Units crosswalk (info lens).}
\[
  S_{\text{bits}} = \frac{\Sent}{\ln 2},
  \qquad
  \text{bits/s} = \frac{\dotSphys}{k_B \ln 2}.
\]

\paragraph{Planck-units checkpoint.}
\[
\begin{aligned}
\ell_P &= \sqrt{\frac{\hbar G}{c^{3}}} && [\mathrm{m}],\\
t_P    &= \sqrt{\frac{\hbar G}{c^{5}}} && [\mathrm{s}],\\
m_P    &= \sqrt{\frac{\hbar c}{G}}     && [\mathrm{kg}],\\
E_P    &= m_P c^{2}
        \;=\; \sqrt{\frac{\hbar c^{5}}{G}} && [\mathrm{J}],\\
T_P    &= \frac{E_P}{k_B}
        \;=\; \sqrt{\frac{\hbar c^{5}}{G}}\,\frac{1}{k_B} && [\mathrm{K}].
\end{aligned}
\]
\noindent\textit{Status:}
\(c\) and \(\hbar\) are fixed by kinematics/phase \cite{einstein1905elektrodynamik};
\(G\) by gravitation; and here \(k_B\) is anchored operationally via the heat–entropy
rate relation \(\dot S_{\mathrm{phys}}=\dot Q/T\) \cite{callen1985thermodynamics}.
With these identifications the temperature scale \(T_P\) is fixed (numerical values from
\cite{codata2018}).

% --------------------------
\section{Temperature as Update Rate}

\paragraph{Conventional view.}
Temperature measures average kinetic energy (in simple models) and serves as the integrating factor for heat \cite{boltzmann1872,callen1985thermodynamics}.

\paragraph{S/t view (rate-normalized entropy update).}
\[
\dotSphys \;=\; \frac{\dot Q}{T}
\qquad\Longleftrightarrow\qquad
T \;=\; \frac{\dot Q}{\dotSphys}.
\]
A hotter region is one with larger \(\dotSphys\).
Equilibrium means \emph{update rates equalize}—the Zeroth Law in rate language.

\paragraph{Tick calibration (link to nats and SI).}
Let a tick carry entropy \(s_{\rm tick}\) (J/K) and last \(\tau\). Then
\[
\dotSphys \;=\; \frac{s_{\rm tick}}{\tau},
\qquad
T \;=\; \frac{\dot Q\,\tau}{s_{\rm tick}}\,,
\qquad
\dot{\Sent}=\frac{\dotSphys}{k_B}.
\]

% --------------------------
\section{From Steam Engines to the Cosmos}

\paragraph{The steam engine.}
Nineteenth-century engines exposed the universality of entropy: some ticks are always consumed irreversibly. The classical laws were codified to describe those constraints \cite{clausius1879mechanical,callen1985thermodynamics}.

\paragraph{The cosmic engine.}
On the largest scales, the \emph{cosmic microwave background (CMB)} encodes the same logic. Early-universe ``ticking'' produced nearly scale-invariant fluctuations with a slight red tilt (\emph{slow-roll estimate} \(n_s\!\approx\!1-2/N\) with \(N\!\sim\!60\); observed \(n_s\!\approx\!0.965\)) \cite{peebles1993principles,planck2018params}.

\paragraph{Punchline (explicit equality under steady conditions).}
Over an interval where \(\dot Q\) and \(T\) are roughly constant,
\[
\Delta \Sphys \;\approx\; \frac{\dot Q}{T}\,\Delta t\,.
\]

% --------------------------
\section{Microstates, Multiplicity, and Statistical Temperature}

\paragraph{(1) Statistical origin.}
A macrostate corresponds to \(\Omega\) microstates:
\[
\Sent \;=\; \ln\Omega\quad\text{(nats)},\qquad
\Sphys \;=\; k_B\,\Sent\quad\text{(J/K)}.
\]
(For bits, divide by \(\ln 2\). Information theory mirrors this structure \cite{shannon1948mathematical}.)

\paragraph{(2) Temperature as sensitivity.}
\[
\frac{1}{T} \;=\; \left(\frac{\partial \Sphys}{\partial U}\right)_V .
\]

\paragraph{(3) Reservoir contact.}
Maximizing total multiplicity of system+\!bath gives \(\delta Q=T\,d\Sphys\): entropy gradients bias energy flow \cite{callen1985thermodynamics}.

\paragraph{(4) Local and global entropy balance.}
Let \(s\) be entropy density (J/K/m\(^3\)) and \(\mathbf j_S\) the entropy flux. Then
\[
\partial_t s \;+\; \nabla\!\cdot\!\mathbf j_S \;=\; \sigma\ \ (\sigma\ge 0),
\qquad
\mathbf j_S\;=\;\frac{\mathbf j_Q}{T}\;+\;s\,\mathbf v\;-\;\sum_i \frac{\mu_i}{T}\,\mathbf j_i,
\]
where \(\mathbf j_Q\) is heat flux, \(\mathbf v\) a bulk velocity, \(\mu_i\) chemical potentials and \(\mathbf j_i\) species fluxes (viscous terms contribute positively to \(\sigma\) and are omitted here for brevity) \cite{callen1985thermodynamics}. Integrated over a volume:
\[
\dot{S}_{\mathrm{phys}} \;=\; \sum_i \frac{\dot Q_i}{T_i} \;+\; \sigma.
\]

% --------------------------
\section{Heat and Internal Energy}

\paragraph{Heat.}
Heat \(Q\) is not a substance; it is transfer—path-dependent tick-generated updates between regions \cite{callen1985thermodynamics}.

\paragraph{Internal energy.}
\(U\) is the stored energy of microscopic updates (motions, configurations, fields)—a state function. Every degree of freedom is a tick source.

% --------------------------
\section{Heat Transfer Mechanisms}

\begin{itemize}
  \item \textbf{Conduction (diffusive updates).}
  Fourier’s law \(\mathbf j_Q=-\kappa\nabla T\) yields nonnegative entropy production:
  \[
    \dot{S}_{\mathrm{phys,cond}}=\int \mathbf j_Q\!\cdot\nabla(1/T)\,dV
    = \int \frac{\kappa\,|\nabla T|^2}{T^2}\,dV \;\ge\; 0\quad(\kappa>0).
  \]
  \item \textbf{Convection (bulk carriage).}
  Parcels move bodily; the entropy-flux contribution is \( \mathbf j_S^{\rm conv} = s\,\mathbf v \).
  \item \textbf{Radiation (ballistic updates).}
  Massless excitations carry updates one hop per tick, bounded by \(c\) \cite{einstein1905elektrodynamik,planck1901law}.
  For \emph{near-equilibrium blackbody-like} emission into vacuum,
  \[
    \dot{S}_{\mathrm{phys,rad}}=\frac{4}{3}\,\frac{P}{T}\,,
  \]
  with departures for non-Planckian spectra or reabsorption.
\end{itemize}

% --------------------------
\section{The Laws of Thermodynamics, Restated}

\subsection*{Zeroth Law}
Equal entropy-update rates \(\Rightarrow\) equal temperature. Thermal equilibrium is tick-rate synchronization.

\subsection*{First Law}
\textbf{Energy is conserved:}
\[
\Delta U \;=\; Q - W.
\]
\paragraph{Work from bias \(\times\) displacement (sign).}
We use \( \Delta U = Q - W\) with \(W\) the work \emph{by} the system:
\[
W \;=\; \int \mathbf F_{\rm by}\!\cdot d\mathbf x,\qquad
P \;=\; \mathbf F_{\rm by}\!\cdot \mathbf v.
\]
(Equivalently, with external work on the system, \(W_{\rm on}=-W\) and \(\Delta U=Q+W_{\rm on}\).)

\subsection*{Clausius inequality}
For any process between fixed end states,
\[
\Delta \Sphys \;\ge\; \int_{t_0}^{t_1} \frac{\delta Q_{\rm rev}}{T},
\]
with equality for reversible paths; for any cycle,
\[
\oint \frac{\delta Q}{T} \;\le\; 0,
\]
with equality if reversible \cite{clausius1879mechanical,callen1985thermodynamics}.

\subsection*{Second Law}
For a \emph{closed system}, the global entropy is non-decreasing; locally, production satisfies \(\sigma\ge 0\).
Entropy can decrease in a \emph{subsystem} only by exporting it via flux \(\mathbf j_S\) \cite{boltzmann1872,shannon1948mathematical,zeh2007arrow}.

\subsection*{Third Law}
If local ticking halts, entropy approaches a constant minimum. For a perfect crystal, this limit is conventionally zero. In the Planck-Cell framing: no ticks, no entropy production.

% --------------------------
\section{Worked Examples}

\subsection*{1) Steam Engine (macro)}
A heat engine takes in \(Q_H\) at \(T_H\) and exhausts \(Q_C\) at \(T_C\).
Carnot efficiency:
\[
\eta \;=\; 1 - \frac{T_C}{T_H}.
\]
Each hot tick carries \(\Delta \Sphys = Q_H/T_H\); running requires dumping \(\Delta S_{\mathrm{phys,C}} = Q_C/T_C\).
The piston is powered by the update-rate difference: an \emph{entropy-tick converter} \cite{callen1985thermodynamics}.

\subsection*{2) A Cup of Water (everyday \(\dotSphys\)-meter)}
Heat \(1~\text{kg}\) of water from \(300\ \text{K}\) to \(310\ \text{K}\) in \(100\ \text{s}\), with \(c_p = 4180\ \text{J/(kg\,K)}\) \cite{crc_handbook}.
\[
\Delta \Sphys \approx m c_p \ln\!\frac{310}{300} \approx 1.37\times 10^{2}\ \text{J/K},\qquad
\dotSphys \approx 1.37 \ \text{J/(K\,s)}.
\]

\subsection*{3) CMB (cosmic)}
Early-universe ticking produced nearly scale-invariant fluctuations with red tilt
\(n_s \approx 1 - 2/N\) for \(N \sim 60\) (\emph{slow-roll} heuristic; observed \(n_s\!\approx\!0.965\)) \cite{peebles1993principles,planck2018params}.

\subsection*{4) The Sun as a Cosmic \(\dotSphys\)-meter}
For near–blackbody radiation at temperature \(T\),
\[
\dot{S}_{\mathrm{phys,rad}} \;=\; \frac{4}{3}\,\frac{P}{T}.
\]
With \(L_\odot \simeq 3.85\times 10^{26}\ \mathrm{W}\) and \(T_\odot \simeq 5.77\times 10^3\ \mathrm{K}\) \cite{allen2000},
\[
\dot{S}_{\mathrm{phys},\odot} \;\approx\; \frac{4}{3}\,\frac{L_\odot}{T_\odot}
\;\approx\; 8.9\times 10^{22}\ \mathrm{J\,K^{-1}\,s^{-1}}.
\]
Absorption by space at \(T_{\rm space}\approx 2.725\ \mathrm{K}\) \cite{planck2018params} gives the net production
\[
\dot{S}_{\mathrm{phys,univ}} \;\approx\; \frac{4}{3}\,L_\odot\!\left(\frac{1}{T_{\rm space}} - \frac{1}{T_\odot}\right)
\;\approx\; 1.9\times 10^{26}\ \mathrm{J\,K^{-1}\,s^{-1}}.
\]
\emph{Bits/s (info lens):}
\[
\frac{\dot{S}_{\mathrm{phys},\odot}}{k_B\ln 2}\ \approx\ 9.3\times 10^{45}\ \text{bits/s},\qquad
\frac{\dot{S}_{\mathrm{phys,univ}}}{k_B\ln 2}\ \approx\ 2.0\times 10^{49}\ \text{bits/s}.
\]
\footnotesize\noindent\emph{Constants used (standard values):} CODATA fundamental constants \cite{codata2018}; solar luminosity and photospheric temperature \cite{allen2000}; water heat capacity \cite{crc_handbook}.
\normalsize

% --------------------------
\section{Interpretation and Foreshadow}
Thermodynamics is not about engines; it is about ticks.
Every law of thermal physics flows from the S/t axiom:
\[
\text{Ticks bias systems toward increasing entropy (\(\sigma\ge 0\)).}
\]
Temperature, heat, and the four laws are systematic bookkeeping for those ticks.
Later we connect the same rate picture to the CMB spectrum, tilt, and polarization—seamlessly from kettles to cosmos \cite{peebles1993principles,planck2018params}. For the broader Planck-Cell program and axioms, see prior articles by the author \cite{langstaff_zenodo_16908311}.

% --------------------------
\section*{Summary box}
\begin{center}
\begingroup
\setlength{\fboxsep}{9pt}% inner padding
\setlength{\fboxrule}{0.6pt}% border thickness
\fbox{%
  \begin{minipage}{0.95\linewidth}
    \setlength{\parindent}{0pt}
    \raggedright\linespread{1.08}\selectfont
    \textbf{Key relations and statements}\par\medskip
    \begin{itemize}
      \setlength{\itemsep}{0.9ex}
      \setlength{\topsep}{0.6ex}
      \setlength{\parsep}{0.4ex}
      \setlength{\partopsep}{0pt}
      \item \textbf{Temperature as update rate:} \(\dotSphys=\dot Q/T \Rightarrow T=\dot Q/\dotSphys\); \(\dot{\Sent}=\dotSphys/k_B\).
      \item \textbf{State vs.\ path:} Heat is path-dependent; internal energy \(U\) is a state function.
      \item \textbf{Microstate entropy:} \(\Sent=\ln\Omega\) (nats), \(\Sphys=k_B\Sent\); information rate: \(\text{bits/s}=\dot{S}_{\mathrm{phys}}/(k_B\ln 2)\).
      \item \textbf{Local balance:} \( \partial_t s+\nabla\!\cdot\!\mathbf j_S=\sigma\), with \( \mathbf j_S=\mathbf j_Q/T + s\,\mathbf v - \sum_i (\mu_i/T)\,\mathbf j_i\), and \(\sigma\ge 0\).
      \item \textbf{Global balance:} \( \dot{S}_{\mathrm{phys}}=\sum_i \dot Q_i/T_i+\sigma\); conduction (\(\kappa>0\)) \(\Rightarrow\ \dot{S}_{\mathrm{phys,cond}}\ge 0\).
      \item \textbf{Radiation (near-BB):} \(\dot{S}_{\mathrm{phys,rad}}=\tfrac{4}{3}\,P/T\) (vacuum; reabsorption/non-Planckian spectra modify this).
      \item \textbf{First Law:} \(\Delta U = Q - W\), with \(W=\int \mathbf F_{\rm by}\!\cdot d\mathbf x\) and \(P=\mathbf F_{\rm by}\!\cdot \mathbf v\).
      \item \textbf{Clausius statements:} \(\Delta \Sphys \ge \int \delta Q_{\rm rev}/T\); \(\oint \delta Q/T \le 0\) (equality if reversible).
      \item \textbf{Meters:} Heat engines as entropy–tick converters; Sun/CMB as near-blackbody \(\dotSphys\)-meters.
    \end{itemize}
  \end{minipage}%
}
\endgroup
\end{center}

% =========================================================
% ===============  BACK MATTER / STATEMENTS  ==============
% =========================================================

\clearpage
\section*{Acknowledgments}
\addcontentsline{toc}{section}{Acknowledgments}
\IfFileExists{other_tex/acknowledgments.tex}{This work is built on the work of countless others. I have not created the pieces myself—I have only arranged them in a way that made sense to me. The real credit belongs to the physicists, mathematicians, and thinkers who discovered, tested, and refined the ideas that form this foundation. I am deeply grateful to those who developed the concepts of entropy, thermodynamics, relativity, quantum mechanics, and information theory—the puzzle pieces this thesis tries to weave into a unifying principle.

I am also thankful for the less visible but equally vital sources of strength that make research possible. The encouragement of peers and mentors, the courage it takes to share an idea, and the faith and love of family all create the conditions in which curiosity can flourish. In those moments of support, difficult work becomes sustainable and creative risks become possible.

I’m grateful, too, for the broader culture of science—for the stories, books, lectures, films, and conversations that circulate ideas. These expressions not only share knowledge but also inspire new connections, allowing interpretations to thrive in ways no single discipline could achieve alone. They are a reminder that science is not only equations on a page but also a human story of exploration and meaning.

I also wish to thank the wider community of researchers whose publications, lectures, and discussions made these ideas accessible. Any originality here lies only in nudging familiar pieces into a new arrangement; the substance belongs to those who created the pieces in the first place.

Finally, I’m grateful for readers willing to explore speculative connections and for the spirit of curiosity that sustains physics itself. Thank you—for leaving a trail of insight to follow. Without your efforts, there would be nothing here to assemble.
}{}

\section*{Competing interests}
\IfFileExists{other_tex/competing_interests.tex}{The author declares no competing interests.
}{}

\section*{Author contributions}
\IfFileExists{other_tex/authors_contributions.tex}{The author solely conceived, developed, analyzed, and wrote the manuscript.
}{}

\section*{Funding}
\IfFileExists{other_tex/funding.tex}{No external funding was received for this work.
}{}

\section*{Data and materials availability}
\IfFileExists{data_accessibility.tex}{This repository hosts the LaTeX sources.
The canonical archive is DOI: \RepoDOI.
Git: \RepoGitHubURL\ (commit \RepoCommit). Version: \PaperVersion.
}{No new data were generated.}

\section*{Disclosure}
\IfFileExists{other_tex/disclosure.tex}{
AI tools were used to assist with drafting and formatting; all analyses, interpretations, and conclusions were verified by the author, who accepts full responsibility for the content.}{}

% ----------------
% Other Works (inserted above bibliography)
% ----------------
\clearpage
\section*{Other Works by the Author}
\addcontentsline{toc}{section}{Other Works by the Author}
The present article on thermal physics in the Planck-Cell medium builds on a sequence
of related works that establish the framework step by step:
\begin{itemize}
  \item \textbf{Temporal Relativity.} C.\,M. Langstaff (2025). Zenodo. DOI: 10.5281/zenodo.17119049.
  \item \textbf{Planck-Cell Kinematics.} C.\,M. Langstaff (2025). Zenodo. DOI: 10.5281/zenodo.17168478.
  \item \textbf{Planck-Cell Mass.} C.\,M. Langstaff (2025). Zenodo. DOI: 10.5281/zenodo.17209646.
  \item \textbf{Planck-Cell Gravity.} C.\,M. Langstaff (2025). Zenodo. DOI: 10.5281/zenodo.17210232.
\end{itemize}
Together, these works define the axioms, kinematic rules, and stiffness
interpretation of mass in the Planck-Cell medium, providing the foundation
on which the present thermal layer is constructed.

% ----------------
% Glossary (prints only if glossaries is loaded)
% ----------------
\IfGlossariesEnabled{
  \clearpage
  \printglossary[title=Glossary]
}

% ----------------
% Bibliography (BibTeX)
% ----------------
\clearpage
\bibliographystyle{plain}
% (No \nocite{*}; only cited items will appear.)
\bibliography{bibliography}

\end{document}
